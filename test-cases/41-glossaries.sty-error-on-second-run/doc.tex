\documentclass{article}

\usepackage[T1]{fontenc}
\usepackage[utf8]{inputenc}

\usepackage{glossaries}

\makeglossaries

% the "name" field in each entry is here to suppress this error:
%
% Package glossaries Error: name key required in \newglossaryentry for entry
 
\newglossaryentry{accented-labél}%forbidden use of active character in glossary label
{
    name=~,
    description={blah}
}

% Note that we could also trigger  the error on the third pdflatex run
% instead of the second, or on the fourth, etc. cf document body.

\newglossaryentry{correct-label}
{
    name=~,
    description={cf \gls{accented-labél}}
}


\newglossaryentry{correct-label2}
{
    name=~,
    description={cf \gls{correct-label}}
}



\begin{document}

The glossaries package: a guide for beginners. Page 2:

\hrule
  As  with  similar  labelling  commands,  such  as  \verb+\label+  or
  \verb+\bibitem+, the label should  not contain active characters, so
  just use  \verb+a, ..., z, A,  ..., Z, 0,  ..., 9+. You may  also be
  able to use some punctuation  characters, unless they have been made
  active (for example, via babel's shorthand activation.)
\hrule

\bigskip
\bigskip

% only one of the three \gls{} lines below should be uncommented:

\gls{accented-labél}%triggers the error on the second pdflatex run

% \gls{correct-label}%triggers the error on the third run
% \gls{correct-label2}%triggers the error on the fourth run

\printglossaries


\end{document}
